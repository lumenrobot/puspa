\subsection*{\textcolor{subsectioncolor}{\textsf{1. \textit{INTRODUCTION}}}}
\addcontentsline{toc}{subsection}{1. \textit{INTRODUCTION}}


Secara umum, perangkat lunak PUSPA dapat dibagi menjadi dua subsistem,
yaitu \textit{server} dan \textit{client}.
Secara khusus, perangkat lunak PUSPA dibagi menjadi modul-modul,
dengan fungsi-fungsinya yang ditunjukkan pada Tabel \ref{FungsiPerangkatLunak}.

\begin{table}
	\centering
		\begin{tabular}{|p{1.8cm}|p{4.7cm}|p{8cm}|}
			\hline
				Subsistem & Modul & Keterangan\\
			\hline
				\textit{Server} & NaturalLanguageAnalyser & mengolah bahasa alami\\
				& NaturalLanguageGenerator & membuat kalimat dari tanggapannya\\
				& DialogueManager & membuat tanggapan dan mengatur arah pembicaraan\\
				& TaskManager & melaksanakan tugas\\
				& KnowledgeBase & berperan sebagai pengetahuan PUSPA\\
				& ServerSocket & membuka \textit{port} dan melakukan komunikasi dengan \textit{client}\\
			\hline
				\textit{Client} & Synthespian & model 3D yang berinteraksi dengan pengguna\\
				& UserInterface & menampilkan \textit{user interface} dan interaksinya\\
				& SpeechSynthesiser & melakukan proses TextToSpeech\\
				& FaceDetector & melakukan proses pengenalan wajah dan \textit{tracking}\\
				& SpeechRecogniser & melakukan pengenalan suara (SpeechToText)\\
				& ClientSocket & melakukan komunikasi dengan \textit{server}\\			
			\hline
		\end{tabular}
\caption{Modul-modul dalam PUSPA}
\label{FungsiPerangkatLunak}
\end{table}
