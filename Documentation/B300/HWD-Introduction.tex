\subsection*{\textcolor{subsectioncolor}{\textsf{1. \textit{INTRODUCTION}}}}
\addcontentsline{toc}{subsection}{1. \textit{INTRODUCTION}}
Bagian ini akan menjelaskan spesifikasi arsitektur sistem dan desain perangkat keras (\textit{hardware}) untuk PUSPA. Secara umum perangkat keras yang digunakan pada setiap penerapan PUSPA mencakup:
\begin{itemize}
	\item sebuah komputer yang bertindak sebagai \textit{server};
	\item satu atau lebih buah komputer yang bertindak sebagai \textit{client}, yang masing-masingnya
dilengkapi dengan:
		\begin{itemize}
			\item sebuah monitor untuk menampilkan wajah PUSPA dan keluaran teks,
			\item sebuah mikrofon untuk berperan sebagai kuping PUSPA,
			\item sebuah \textit{speaker} untuk berperan sebagai mulut PUSPA,
			\item sebuah \textit{webcam} untuk berperan sebagai mata PUSPA,
			\item sebuah kibor dan tetikus untuk menerima masukan teks,
		\end{itemize}
	\item dan sebuah layar \textit{multitouch} sebagai tambahan keluaran wajah dan teks PUSPA.
\end{itemize}
