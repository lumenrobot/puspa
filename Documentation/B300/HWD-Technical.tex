\subsection*{\textcolor{subsectioncolor}{\textsf{5. \textit{TECHNICAL SPECIFICATION}}}}
\addcontentsline{toc}{subsection}{5. \textit{TECHNICAL SPECIFICATION}}

Adapun spesifikasi perangkat keras secara umum, ditunjukkan oleh Tabel \ref{spesifikasi}.
\begin{table}
	\centering
		\begin{tabular}{|p{4cm}|p{11cm}|}
			\hline
				Perangkat & Spesifikasi\\
			\hline
				Komputer \textit{server} & Prosesor : AMD Athlon XP 5000+ \\
				& RAM : 2 GB\\
				& OS : Debian stable\\
			\hline
				Komputer \textit{client} & Prosesor : AMD Athlon XP 5000+\\
				& RAM : 4 GB\\
				& OS : Windows 7 Ultimate 64bit\\
				&\\
				& Monitor : L1753S\\
				&\\
				& Mikrofon : Logitech Webcam Pro 9000\\
				&\\
				& Webcam	: Logitech Webcam Pro 9000\\
				&\\
				& Speaker : Altel Lansing VS4121\\
				&\\
				& Keyboard \& mouse : Microsoft Standard Keyboard\&Mouse\\
			\hline
				Komputer \textit{client} & Apple Mac\\
			\hline
				Layar \textit{multitouch} & Proyektor : Benq\\
				& Layar : Tagscreen 50''\\
			\hline
		\end{tabular}
\caption{Spesifikasi}
\label{spesifikasi}
\end{table}
