\subsection*{\textcolor{subsectioncolor}{\textsf{1. \textit{INTRODUCTION}}}}
\addcontentsline{toc}{subsection}{1. \textit{INTRODUCTION}}
\hyphenation{meng-gan-ti-kan de-ngan u-cap-an peng-u-cap-an mi-sal-nya}

%Deskripsi umum mengenai konsep sistem yang akan diimplementasikan
Proyek PUSPA bertujuan untuk menciptakan sebuah \textit{chatterbot}, sebuah \textit{synthespian}, dan sebuah asisten dalam satu kemasan.


\subsubsection*{\textit{Purpose}}

Tujuan dari mengembangkan PUSPA adalah untuk ikut mewujudkan gagasan akan mampunya mesin menggantikan manusia dalam menyelesaikan pekerjaan-pekerjaan.


\subsubsection*{\textit{Scope}}

Karena tujuan proyek ini adalah menciptakan sesuatu yang menyimulasikan manusia, kata kerja yang digunakan untuk menjelaskan perilaku PUSPA pada dokumen ini merupakan kata kerja yang biasa digunakan dengan manusia sebagai pelakunya.

PUSPA menggunakan bahasa Inggris sebagai bahasa utama interaksi dengan penggunanya.
Oleh karena itu, yang dimaksud dengan kata ``bahasa'' dalam dokumen ini adalah bahasa Inggris, kecuali disebutkan bahwa yang dimaksud adalah bahasa lain.

Masukan dan keluaran PUSPA adalah simulasi percakapan dalam bentuk teks dan/atau suara.
Oleh karena itu, yang dimaksud dengan kata ``pembicaraan'', ``percakapan'', atau kata lain semacamnya dalam dokumen ini mencakup yang dilakukan dalam bentuk teks dan/atau suara.

Pengembangan PUSPA pada tahap awal lebih ditekankan pada perangkat lunaknya.
Dengan kata lain, penerapan perangkat keras untuk mewujudkan fungsionalitas PUSPA adalah dengan menggunakan komputer umum beserta perangkat tambahan yang juga umum.
Hal ini berarti apapun yang PUSPA dapat laksanakan bersifat terbatas pada apa yang komputer umum mungkin laksanakan, dan interaksi apapun dengan penggunanya adalah interaksi yang mungkin terjadi antara komputer dan penggunanya.


\subsubsection*{\textit{Definitions}}

\begin{description}
\item[Chatterbot] Program komputer yang dirancang untuk berhubungan dengan manusia dengan menyimulasikan percakapan melalui suara atau tulisan.
\item[Face detection] Teknologi komputer yang menentukan kedudukan dan ukuran wajah manusia pada citra digital.
\item[Natural language processing] Bidang khusus pada ilmu komputer dan ilmu bahasa tentang hubungan antara komputer dan bahasa alami manusia.
\item[Puspa] Sebuah nama yang cukup umum dipakai oleh orang Indonesia, yang menjadi asal nama proyek ini.
\item[Speech recognition] Kemampuan komputer dalam mengenali suara yang dihasilkan oleh ucapan manusia.
\item[Speech synthesis] Pembuatan ucapan manusia tiruan.
\item[Speech-to-text] Sistem yang menciptakan bentuk tulisan pada berkas elektronik dari suatu pengucapan bersuara.
\item[Synthespian] Karakter makhluk hidup buatan yang dibangkitkan oleh komputer.
\item[Text-to-speech] Sistem yang menciptakan bentuk pengucapan bersuara dari suatu tulisan pada berkas elektronik.
%\item[Voice recognition] Bagian dari \textit{speech recognition} yang lebih diarahkan untuk mengenali siapa pembicaranya.
\end{description}


\subsubsection*{\textit{Overview}}

PUSPA muncul dalam bentuk gambar 3D kepala manusia pada layar monitor komputer pengguna.
Pengguna dapat berbicara dengan PUSPA dalam bahasa sehari-hari. Pengguna juga dapat meminta PUSPA untuk melaksanakan pekerjaan yang dapat dilakukan dengan komputer, misalnya seperti menulis surel dan mengirimkannya.
