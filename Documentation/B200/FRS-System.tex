\subsection*{\textcolor{subsectioncolor}{\textsf{2. \textit{SYSTEM FUNCTIONS}}}}
\addcontentsline{toc}{subsection}{2. \textit{SYSTEM FUNCTIONS}}
\hyphenation{PUSPA me-nge-nal-i a-wal}
%Deskripsi fungsi sistem secara umum


\subsubsection*{\textit{Product perspective}}

Pengembangan \textit{conversational agent} di dunia sudah dimulai berpuluh-puluh tahun yang lalu, sehingga sudah ada banyak produk yang berfungsi seperti PUSPA.
Bagaimanapun, produk-produk tersebut tidak sepenuhnya sama dengan PUSPA.

%Dari segi bahasa, setidaknya sudah ada penerapan untuk bahasa Inggris, Jerman, Spanyol, Prancis, Itali, dan Portugis.
%Di Indonesia sendiri sebenarnya penelitian di bidang seperti NLP sudah dilakukan, namun belum ada yang mengarah ke pengembangan produk seperti PUSPA dan berhasil.
%Di samping itu, karena penggunaan PUSPA melibatkan banyak interaksi, PUSPA diharapkan dapat mempengaruhi penggunanya untuk ikut menggunakan bahasa secara baik dan betul.
%Hal ini akan memberi dampak yang baik bagi bahasa Indonesia itu sendiri, yang penggunaannya telah banyak disimpangkan oleh masyarakat.

Dari segi kepribadian, ada sebagian dari produk-produk sebelumnya yang mungkin membuat penggunanya tersinggung atau tertekan, misalnya karena memang dirancang untuk berpendapat tajam, atau terpengaruh oleh penggunanya yang memang sengaja menyalahgunakannya dengan memberi masukan bahasa yang buruk.
PUSPA dirancang untuk berkepribadian yang lebih halus, dan dapat menjaga diri dari pengaruh buruk penggunanya.


\subsubsection*{\textit{Product functions}}

PUSPA berfungsi sebagai teman bicara, dan sebagai asisten yang dapat diperintah.


\subsubsection*{\textit{User characteristics}}

Pada penerapannya, yang digunakan adalah komputer.
Walaupun begitu, pengguna PUSPA tidak perlu memiliki pengetahuan yang dalam tentang komputer, karena PUSPA dirancang untuk dapat mengerti bahasa alami penggunanya.
Untuk dapat melakukan percakapan dengan PUSPA, kemampuan yang harus dimiliki oleh pengguna hanyalah kemampuan berbicara saja.
Bagaimanapun, disebabkan oleh hal yang sama, untuk dapat memanfaatkan PUSPA sebagai asisten, pengguna setidaknya harus memiliki pengetahuan akan apa yang dapat dilakukan dengan komputer, seperti surat-menyurat atau mencari informasi.


\subsubsection*{\textit{General constraints}}

Tidak semua kemampuan manusia yang mendukung pembicaraan (seperti kemampuan mengenali lawan bicara melalui penglihatan) dapat ditiru oleh PUSPA pada pengembangan tahap awal.
Di samping itu, pengembangan PUSPA mengalami hambatan yang mungkin dialami dalam menggunakan komputer pada umumnya, seperti dibutuhkannya usaha lebih agar PUSPA dapat dijalankan di berbagai macam sistem operasi.


\subsubsection*{\textit{Assumptions and dependencies}}

Pengguna dianggap bersedia berbicara dengan benda mati dan bersedia dibantu dalam mengerjakan tugas-tugas.
