\subsection*{\textcolor{subsectioncolor}{\textsf{2. \textit{SW SYSTEM FUNCTIONS}}}}
\addcontentsline{toc}{subsection}{2. \textit{SW SYSTEM FUNCTIONS}}
\hyphenation{di-sam-pai-kan}

Berikut ini adalah penjelasan fungsi-fungsi yang dimiliki sistem.
\begin{itemize}
\item Menerima masukan teks\\
  Sebuah kotak disediakan sebagai bagian dari GUI PUSPA,
  agar pengguna dapat berbicara dengan PUSPA dalam bentuk teks.
\item Menerima masukan suara\\
  Selain dalam bentuk teks, pengguna dapat juga berbicara dengan PUSPA dalam bentuk suara.
\item Menerima masukan citra\\
  PUSPA dibuat seakan-akan dapat melihat penggunanya,
  dengan menangkap kedudukan wajah pengguna dan menggunakan datanya untuk mengarahkan mata PUSPA.
\item Memberikan keluaran teks\\
  Untuk menampilkan tanggapan PUSPA terhadap masukan teks, tempat bagi tanggapan teksnya disediakan juga.
\item Memberikan keluaran suara\\
  Tanggapan PUSPA yang masih dalam bentuk teks diolah sehingga mengeluarkan suara yang mewakili tanggapan tersebut.
\item Memberikan keluaran citra\\
  PUSPA tampil di hadapan pengguna dalam bentuk gambar 3D manusia.
\item Mengolah masukan percakapan\\
  PUSPA mengolah percakapan yang dilakukan penggunanya, sehingga maksud yang disampaikan pengguna dapat dipahami oleh PUSPA.
\item Membuat simulasi percakapan\\
  Tanggapan PUSPA terhadap masukan pengguna, dikemas dalam bentuk percakapan.
\item Meluncurkan program lain untuk melaksanakan tugas\\
  PUSPA dapat diperintah untuk melaksanakan hal-hal yang dapat pengguna komputer lakukan dengan komputernya, seperti mengirim surel.
\item Menyimpan data\\
  PUSPA juga dapat digunakan untuk menyimpan data, yang kemudian dapat dicari lagi dengan menggunakan bahasa manusia.
\end{itemize}
