\subsection*{\textsf{\normalsize 2.10\hspace{0.5cm}PERHITUNGAN \textit{NET PRESENT VALUE} (NPV)}}
\addcontentsline{toc}{subsection}{2.10 PERHITUNGAN \textit{NET PRESENT VALUE}}

Dalam Lampiran C hingga F dilampirkan perhitungan aliran kas (\textit{cashflow}). Dari data pada Lampiran F didapat \textit{Net Cash Flow} per tahun. Dengan asumsi \textit{discount rate} sebesar 15\% didapatkan NPV dengan perhitungan seperti pada Tabel~\ref{tab:npv}.

Nilai NPV yang didapat adalah sebesar Rp 4.137.265.900. Dengan NPV $>$ 0, dapat disimpulkan bahwa pengembangan PUSPA adalah layak usaha.

\begin{table}
	\centering
		\begin{tabular}{|>{\scriptsize}p{1.4cm}|>{\scriptsize}r|>{\scriptsize}r|>{\scriptsize}r|>{\scriptsize}r|>{\scriptsize}r|>{\scriptsize}r|}
\hline
\textbf{Tahun Ke-} & \textbf{0} & \textbf{1} & \textbf{2} & \textbf{3} & \textbf{4} & \textbf{5}\\
\hline
\multicolumn{7}{c}{}\\
\hline
\textbf{Interest Rate} & 15,00\% & \multicolumn{5}{c|}{}\\
\hline
\textbf{Discounted Factor} & 1,00 & 0,87 & 0,76 & 0,66 & 0,57 & 0,50\\
\hline
\textbf{NCF} & Rp100.000.000 & Rp311.850.000 & Rp855.720.000 & Rp1.384.092.000 & Rp1.847.092.000 & Rp2.720.092.000\\
\hline
\textbf{PV} & Rp100.000.000 & Rp271.173.913 & Rp647.047.259 & Rp910.062.957 & Rp1.056.366.723 & Rp1.352.615.049\\
\hline
\textbf{NPV} & Rp4.137.265.900 & \multicolumn{5}{c|}{}\\
\hline
\end{tabular}

	\caption{Tabel perhitungan NPV}
	\label{tab:npv}
\end{table}
