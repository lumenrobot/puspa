\subsection*{\textsf{\normalsize 2.5\hspace{0.5cm}UPAYA PENGEMBANGAN}}
\addcontentsline{toc}{subsection}{2.5 UPAYA PENGEMBANGAN}
\hyphenation{de-ngan Speech-Recogniser}

Riset dan pengembangan PUSPA dilakukan oleh kelompok tesis yang saat ini anggotanya adalah:
\begin{itemize}
\item Erik Prabowo Kamal,
%\item Rio Andita Setiabakti,
\item dan Willy Derbyanto,
\end{itemize}
dengan kemungkinan penambahan anggota seiring dengan waktu dan kebutuhan, dan dengan pengawasan tim dan pembimbing tesis sebagai tim ahli.

Alat pengembangan yang digunakan adalah sebagai berikut:
\begin{itemize}
	\item Bahasa pemrograman: C++.
	\item Pustaka/\textit{Framework}: C++ Std Lib, Boost, OpenCog.
	\item \textit{Version control system}: Git.
	\item Dokumentasi: \LaTeX.
	\item Alat lainnya: CMake.
\end{itemize}

Pengembangan PUSPA dibagi menjadi tiga tahap:
\begin{itemize}
	\item Tahap I, tahap pengembangan fungsi dasar yang meliputi fungsi percakapan, dan fungsi yang berkaitan dengan rupa (\textit{modeling}, animasi dasar, dan \textit{rendering}).
	\item Tahap II, tahap pengembangan fungsi unggulan yang meliputi fungsi pelaksanaan tugas, mata dan kamera, kepribadian, animasi lanjut, dan pelacakan wajah pengguna.
	\item Tahap III, yaitu tahap pengembangan fungsi lanjut seperti hubungan \textit{client/server} fitur \textit{ubiquity}.
\end{itemize}
Ketiga tahapan tersebut dilaksanakan selama sembilan bulan. Tabel~\ref{tab:DevTime} menunjukkan tabel pembagian tugas dan waktu pengerjaannya.

\begin{table}
	\centering
		\begin{tabular}{|>{\small}l|>{\small}c|>{\small}p{4cm}|>{\small}c|>{\small}p{2cm}|}
		\hline
			\textbf{Proses\slash \textit{Task}} & 		\textbf{Tahap} & \textbf{\textit{Deliverables}} & \textbf{Jadwal} & \textbf{Kebutuhan Sumber Daya}\\
		\hline
			\multicolumn{5}{|>{\small}c|}{\textbf{\textit{Synthespian}}}\\
		\hline
			\textit{Modeling} & I & Model 3D kepala - torso lengkap dengan \textit{skeleton}-nya & 01/10/09 - 31/10/09 & 1 orang\\
		\hline
			Animasi dasar & I & Animasi-animasi dasar dari \textit{synthespian} & 02/11/09 - 28/11/09 & 1 orang\\
		\hline
			\textit{Real-time rendering} & I & Menampilkan \textit{synthespian} dalam \textit{engine} & 02/11/09 - 28/11/09 & 1 orang\\
		\hline
			Animasi kepribadian & II & Animasi berdasarkan sensor-sensor & 01/12/09 - 13/02/10 & 1 orang\\
		\hline
			Animasi vokal & II & Animasi vokal untuk berbicara & 16/02/10 - 30/04/10 & 1 orang\\
		\hline
			\multicolumn{5}{|>{\small}c|}{\textbf{Sensor dan Suara}}\\
		\hline
	Sensor penglihatan & II & Modul FaceDetector untuk mengenali wajah pengguna & 01/12/09 - 30/04/10 & 1 orang\\
		\hline
			Suara & II & Modul SpeechSynthesiser menyuarakan keluaran teks & 01/12/09 - 15/01/10 & 1 orang\\
		\hline
			Sensor pendengaran & II & Modul SpeechRecogniser mengenali perintah suara & 15/01/10 - 30/04/10 & 1 orang\\
		\hline
			\multicolumn{5}{|>{\small}c|}{\textbf{Antarmuka \textit{Soft Machine}}}\\
		\hline
			\textit{User Interface} & II & UI dengan \textit{soft control} & 01/02/10 - 27/02/10 & 1 orang\\
		\hline
			Layar & II & Layar multi-sentuh dengan 		\textit{holoscreen} & 01/03/10 - 30/04/10 & 1 orang\\
		\hline
			\multicolumn{5}{|>{\small}c|}{\textbf{\textit{Server}}}\\
		\hline
			\textit{Chatterbot} & I & Dapat menanggapi obrolan teks & 01/02/10 - 28/02/10 & 1 orang\\
		\hline
			Asisten & II & Dapat menanggapi tugas sebagai asisten: jadwal, tugas, dan \textit{to do} & 01/03/10 - 30/04/10 & 1 orang\\
		\hline
			\multicolumn{5}{|>{\small}c|}{\textbf{\textit{Client/Server}}}\\
		\hline
			\textit{Client-Server Dev.} & III & \textit{Client/server} untuk masing-masing terminal & 03/05/10 - 30/06/10 & 1 orang\\
		\hline
			\multicolumn{5}{|>{\small}c|}{\textbf{\textit{Finishing}}}\\
		\hline
			\textit{Bug fixing} & III & Pencarian \textit{bug} dan penyempurnaan produk & 03/05/10 - 30/06/10 & 1 orang\\
		\hline
	\end{tabular}
	\caption{Tugas dan jadwal pengembangan PUSPA}
	\label{tab:DevTime}
\end{table}
