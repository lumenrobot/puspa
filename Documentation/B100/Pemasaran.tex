\subsection*{\textsf{\normalsize 2.9\hspace{0.5cm}UPAYA PEMASARAN}}
\addcontentsline{toc}{subsection}{2.9 UPAYA PEMASARAN}
\hyphenation{me-ngem-bang-kan mem-ben-tang-kan}

Dikarenakan produk PUSPA masih berada dalam masa terobosan teknologi, langkah pemasaran yang cocok adalah:
\begin{enumerate}
	\item pembahasan teknologi,
	\item \textit{launching} produk pada \textit{event}-\textit{event} tertentu, dan
	\item demonstrasi di depan publik.
\end{enumerate}
Ketiga langkah di atas bertujuan untuk menarik perhatian media massa dan industri, sehingga mereka tertarik untuk membeli. Ketiganya dapat dilakukan dalam satu \textit{event}, yaitu pameran.

Pemasaran PUSPA harus dilakukan melalui pameran-pameran teknologi. Berikut ini adalah strategi yang setidaknya harus dilakukan untuk memasarkan PUSPA di Indonesia.
\begin{itemize}
	\item Mengikutsertakan PUSPA dalam lomba-lomba teknologi dan bisnis, misalnya INAICTA.
	\item Mengikutsertakan PUSPA dalam pameran-pameran yang diadakan oleh ITB, misalnya Digital Media dan Game Festival.
	\item Mengikutsertakan PUSPA dalam pameran-pameran Departemen Kominfo dan Ristek, misalnya pameran RISTI.
	\item Mengikutsertakan PUSPA dalam pameran-pameran industri, misalnya FGDExpo.
\end{itemize}
Semua pameran tersebut di atas dilaksanakan setiap tahun di Indonesia.

Selain keikutsertaan dalam berbagai pameran, tim pemasaran PUSPA pun mengembangkan jaringan bisnis dengan menjadi anggota komunitas tertentu misalnya Bandung High Tech Valley, Bandung Creative City, dan lain sebagainya. Hal ini diperlukan untuk membentangkan jaringan bisnis dan pengetahuan akan perusahaan pengembang PUSPA. Tim pemasaran pun perlu mengikuti aktivitas inkubasi-inkubasi bisnis yang mulai marak dan tumbuh kembang di Indonesia.
