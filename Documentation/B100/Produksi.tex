\subsection*{\textsf{\normalsize 2.6\hspace{0.5cm}UPAYA PRODUKSI}}
\addcontentsline{toc}{subsection}{2.6 UPAYA PRODUKSI}
\hyphenation{pe-ngem-bang-an}

Produksi PUSPA dapat dilaksanakan setelah purwarupa atau unit demonstrasi selesai dikembangkan. PUSPA tidak dikembangkan secara masal, namun menurut permintaan. Artinya PUSPA dipasarkan dengan unit demo di mana calon pengguna akan menyesuaikan PUSPA dengan kebutuhan pembeliannya. Oleh karena itu diperlukan tim produksi yang terpisah dengan tim riset dan pengembangan. Tim produksi memfokuskan diri pada kostumisasi PUSPA sesuai permintaan pembeli.

Kebutuhan bengkel utamanya adalah untuk aspek estetika terminal PUSPA, namun dapat pula produksi terminal khususnya integrasi layar dan CPU dialihdayakan ke pihak ketiga (\textit{outsourcing}). Hal ini dikarenakan investasi bengkel jauh lebih mahal ketimbang ongkos alih daya per unitnya, padahal yang dibutuhkan justru adalah komputer dan perangkat lunak pengembangan maupun produksi. 

PUSPA memiliki peluang keberhasilan di pasar, mengikuti analisis Gartner tahun 2009 tentang \textit{Hype Cycle of Emerging Technologies 2009} yang akan dijelaskan dalam analisis pasar, dengan \textit{market share} yang masih terbuka luas. Dengan asumsi penjualan sepuluh buah saja pada produksi tahun pertama, produk telah mencapai \textit{Break Event Point} (BEP).

Tabel~\ref{tab:ProdTime} menunjukkan ilustrasi mulai dari perencanaan dengan pembeli hingga \textit{delivery}.

\begin{table}
	\centering
		\begin{tabular}{|l|c|p{3.5cm}|c|p{3.5cm}|}
		\hline
			\textbf{Proses\slash \textit{Task}} & 		\textbf{Tahap} & \textbf{\textit{Deliverables}} & \textbf{Jadwal} & \textbf{Kebutuhan Sumber Daya}\\
		\hline
			\textit{Brainstorming} & I & Berita Acara Pertemuan desain PUSPA & 1-2 hari & 1 orang pemasaran\\
		\hline	
			Kesepakatan Kerjasama & I & MoU dengan pembeli & 1-2 hari & 1 orang pemasaran\\
		\hline
			Produksi PUSPA & II & Perangkat lunak siap di-\textit{install} ke terminal & 2 bulan & 3 orang pengembang\\
		\hline
			Produksi Terminal & II & Terminal yang siap di-\textit{deliver} & 2 minggu & 2 orang\\
		\hline
	\end{tabular}
	\caption{Ilustrasi jadwal produksi PUSPA}
	\label{tab:ProdTime}
\end{table}
