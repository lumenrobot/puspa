\subsection*{\textsf{\normalsize 1.2\hspace{0.5cm}TUJUAN PENULISAN DAN APLIKASI\slash KEGUNAAN DOKUMEN}}
\addcontentsline{toc}{subsection}{1.2 TUJUAN PENULISAN DAN APLIKASI\slash KEGUNAAN DOKUMEN}
\hyphenation{meng-i-kut-i}

%Tujuan/maksud penulisan dokumen ini, dan ditujukan kepada siapa
Tujuan utama tulisan ini dibuat adalah agar para penulis dokumen ini, yang sedang mengikuti Program Magister Teknik Elektro Opsi Teknologi Media Digital dan Game di ITB, dapat memenuhi salah satu dari persyaratan-persyaratan pembuatan produk yang nantinya akan menjadi bahan tesis.
Tulisan ini ditujukan kepada tim dan pembimbing tesis LSKK STEI--ITB.

%Aplikasi/kegunaan dokumen
Dokumen ini adalah dokumentasi pendahuluan, dengan kata lain menjadi tempat diungkapkannya gagasan awal akan produk yang dibuat.
Keputusan akan kelayakan proyek yang diajukan dapat diambil berdasarkan hal-hal yang dipaparkan dalam dokumen ini, seperti spesifikasi produk yang dituju, sumber daya yang ada, rencana pelaksanaan, hasil yang mungkin didapatkan, dan sebagainya.
