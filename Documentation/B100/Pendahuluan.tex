\subsection*{\textsf{\normalsize 2.1\hspace{0.5cm}PENDAHULUAN}}
\addcontentsline{toc}{subsection}{2.1 PENDAHULUAN}
\hyphenation{PUSPA da-pat di-sing-kat de-ngan tam-pil-an me-ma-in-kan}

Bagaimana seandainya kita memiliki asisten yang siap siaga 24 jam untuk mengatur jadwal, tugas, dan apa yang harus dikerjakan? Bagaimana rasanya memiliki teman \textit{virtual} yang dapat membantu pekerjaan sehari-hari, dan bagaimana rasanya berkomunikasi secara \textit{audiovisual} dengan komputer pribadi? Produk kami menjawab itu semua.

%Penggambaran umum mengenai konsep sistem/produk
Konsep produk ini berangkat dari konsep yang terdapat pada \textit{conversational agent}, yaitu gagasan akan mampunya komputer melakukan perbincangan dengan manusia, dalam bahasa manusia.
Di samping dapat mengolah bahasa alami, sistem ini juga dirancang untuk memainkan peran tertentu, dan memiliki suatu watak, kepribadian, dan ciri-ciri tertentu, yang menjadikannya suatu \textit{synthetic actor}, yang juga dikenal dengan istilah \textit{synthetic thespian} (yang biasa disingkat menjadi \textit{synthespian}).

%Fitur-fitur dasar
Fitur dasar dari sistem ini adalah sistem ini dirancang untuk mampu melakukan perbincangan dengan penggunanya dalam bahasa Inggris sebagai bahasa utama.
Di samping itu, produk ini diarahkan untuk berperan sebagai asisten, sehingga sistem ini juga dirancang untuk mampu memahami permintaan penggunanya dan berusaha mewujudkannya, dengan syarat, permintaan tersebut masih dalam ruang lingkup pekerjaan yang mungkin diselesaikan oleh sistem ini.

%Fitur-fitur unggulan
Fitur unggulannya adalah produk ini memiliki tampilan 3 dimensi, yang dilengkapi dengan kemampuan dilakukannya \textit{morphing} pada gambar otot-otot wajah karakternya, sehingga dapat menggambarkan ekspresi wajah dan membuat pengalaman pengguna menjadi lebih menarik.
Kepribadian yang diberikan kepada sistem adalah kepribadian yang berpengertian terhadap penggunanya, sehingga pengguna tidak akan merasa tersinggung ataupun tertekan ketika berbincang dengan sistem, dan juga agar sistem ini tidak menjadi suatu sistem yang membahayakan penggunanya seperti halnya karakter HAL 9000 pada film Stanley Kubrick yang berjudul \emph{2001: A Space Odyssey}.

Berlatarbelakangkan konsep dan fitur-fitur yang telah dijelaskan, produk ini dinamakan ``PUSPA'', yang merupakan singkatan rekursif dari ``{\scshape Puspa}'s an Understanding Synthespian that Provides Assistance''.

%Tambahan dari Rio
Selain tampilan 3 dimensi, interaksi manusia-komputer PUSPA dikemas dalam bentuk \textit{soft machine}. \textit{Soft machine} diterapkan dengan menggunakan \textit{soft control} yang tampilkan secara \textit{real-time} dan layar multi-sentuh (\textit{multi-touch screen}) untuk interaksinya. Perangkat multi-sentuh yang dikembangkan menggunakan prinsip \textit{Frustrated Total Internal Reflection} (FTIR) yang diperkenalkan oleh Jeff Han. Layarnya sendiri menggunakan \textit{holoscreen} yaitu \textit{rear projection film} yang transparan.

Pada penerapan PUSPA, yang disebut sebagai terminal adalah satu kesatuan \textit{client} yang terdiri dari CPU dan perangkat layar tampilan. PUSPA terdiri dari beberapa terminal yang ditempatkan di ruangan yang berbeda-beda. PUSPA akan teraktivasi (ditandai dengan kemunculan \textit{synthespian}) jika pemilik PUSPA dikenali melalui \textit{face recognition}. Sebuah \textit{server} mengatur di ruangan mana PUSPA harus teraktivasi.

Tiga faktor menjadi dasar pemikiran pengembangan PUSPA. Pertama, alangkah asyiknya jika kita dapat berbincang-bincang dengan komputer di rumah. Kedua, alangkah menyenangkannya hidup jika komputer benar-benar menjadi asisten pribadi kita. Ketiga, alangkah nyaman jika asisten komputer pribadi kita mengikuti ke mana saja. Dengan pemasaran yang tidak membutuhkan biaya besar karena PUSPA tergolong teknologi terobosan, nilai penjualan sepuluh unit saja per tahun sudah dapat memberikan keuntungan yang berkali lipat. Selanjutnya kami berharap apa yang diajukan dalam dokumen ini dapat dipertimbangkan untuk diwujudkan.
