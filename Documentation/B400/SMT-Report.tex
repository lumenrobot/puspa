\subsection*{\textcolor{subsectioncolor}{\textsf{5. \textit{TEST REPORT}}}}
\addcontentsline{toc}{subsection}{5. \textit{TEST REPORT}}


\subsubsection*{ServerSocket \& ClientSocket}

Kembali pada Gambar \ref{SocketDegenerate},
yang terjadi saat \textit{server} dijalankan tetapi \textit{client} tidak dijalankan adalah \textit{server} tetap menunggu, yang sebenarnya adalah kejadian yang bukan merupakan galat.
Di lain pihak, yang terjadi saat \textit{client} dijalankan tetapi \textit{server} tidak dijalankan adalah program \textit{client} langsung berhenti karena tidak banyak yang \textit{client} dapat lakukan tanpa \textit{server}.

Kembali pada Gambar \ref{SocketBoundary},
yang terjadi saat karakter yang banyak melebihi ukuran penyangga yang tersedia di \textit{server} adalah terpotongnya data yang diterima.
Ternyata datanya diterima semua, hanya setelah dikirim kembali ke \textit{client},
yang diterima hanya potongan pertama.

Kembali pada Gambar \ref{SocketNonUnique},
jika datanya bersifat normal maka tidak ada galat yang muncul yang bersangkutan dengan pengiriman dan penerimaan data.

Walaupun sudah dapat berjalan dan diandalkan,
masih terdapat \textit{bug} yaitu penampilan alamat IP yang tidak betul.
Terlihat pada Gambar \ref{ClientSocketBoundary} dan Gambar \ref{ClientSocketNonUnique},
alamat IPv4 \textit{server} yang ditampilkan pada \textit{client} sudah hampir betul,
hanya bagian tengahnya yang tidak terbaca.
Di lain pihak, seperti terlihat pada Gambar \ref{ServerSocketBoundary} dan Gambar \ref{ServerSocketNonUnique},
alamat IPv6 \textit{client} yang ditampilkan pada \textit{server} sama sekali tidak mendekati.

Yang disarankan sebagai perbaikan pada ServerSocket adalah kemampuan membuat sambungan dan berkomunikasi dengan lebih daripada satu \textit{client},
kemampuan untuk melakukan pengiriman tanpa harus berpasangan dengan penerimaan terlebih dahulu atau sebaliknya,
dan kemampuan mengumumkan layanannya dengan \textit{zeroconf}.

Sama seperti ServerSocket, hal pertama yang disarankan sebagai perbaikan pada ClientSocket adalah kemampuan untuk melakukan pengiriman tanpa harus berpasangan dengan penerimaan terlebih dahulu atau sebaliknya.
Di samping itu, melengkapi saran untuk ServerSocket berkenaan dengan \textit{zeroconf},
ClientSocket disarankan memiliki kemampuan merambah layanan yang diumumkan \textit{server} dengan \textit{zeroconf},
dan menyerahkan keputusan kepada pengguna apakah akan menyambung ketika layanannya ditemukan atau apakah akan keluar dari program ketika layanannya tidak ditemukan.


\subsubsection*{DialogueManager}

Kembali pada Gambar \ref{DialogueManagerDegenerate},
yang terjadi saat masukan yang diberikan maknanya tidak dimengerti oleh modul ini adalah dinyatakannya bahwa PUSPA tidak mengerti apa yang dimaksud oleh pengguna,
yang merupakan tanggapan \textit{default}.

Kembali pada Gambar \ref{DialogueManagerNonUnique},
jika datanya bersifat normal maka hasilnya dapat berupa suatu percakapan yang memiliki tujuan tertentu,
walaupun kalimat yang dimasukkan dan dihasilkan masih terbatas karena belum digunakannya dua modul pendukung modul ini,
yaitu NaturalLanguageAnalyser dan NaturalLanguageGenerator.

Yang disarankan sebagai perbaikan pada modul ini adalah kemampuan menghubungkan setiap masukan dengan konteks percakapan,
misalnya jika pengguna langsung menyuruh tanpa menyapa terlebih dahulu,
maka modul ini akan memastikan dahulu apakah pengguna sedang berbicara dengan PUSPA.
Saran berikutnya adalah pemeriksaan akan sah atau tidaknya masukan pengguna.
Saran terakhir untuk saat ini adalah penerapan unsur-unsur lain dalam \textit{information state},
seperti \textit{user model},
misalnya dengan membuat PUSPA akan menuruti permintaan hanya jika penggunanya memiliki wewenang,
yang tentu saja modul ini harus dapat mengenali penggunanya terlebih dahulu dengan suatu autentikasi.
