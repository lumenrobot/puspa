\subsection*{\textcolor{subsectioncolor}{\textsf{3. \textit{PROGRAMMING PROCEDURE}}}}
\addcontentsline{toc}{subsection}{3. \textit{PROGRAMMING PROCEDURE}}


\subsubsection*{ServerSocket  \& ClientSocket}

Kode utama yang tercakup dalam modul ServerSocket dan ClientSocket masing-masing dapat ditemukan pada berkas \texttt{ServerSocket.cpp} dan \texttt{ClientSocket.cpp},
yang masing-masingnya memiliki berkas \textit{header}.
Untuk menguji modul-modul ini, telah disediakan juga masing-masing berkas \texttt{serversockettest.cpp} dan \texttt{clientsockettest.cpp}.
Semua kode yang telah disebutkan di atas membutuhkan berkas \texttt{Errors.h} dan \texttt{Errors.cpp} dari direktori \texttt{common} karena sudah mengandung antisipasi terhadap galat.

Pada tahap awal ini,
sambungan yang dibangun masih dibatasi dengan satu \textit{client} saja setiap waktu saja (pada peubah \texttt{backLog}).

Pertama, alamat dan port \textit{server} ditentukan pada kedua modul,
informasi tentang alamatnya didapatkan melalui fungsi \texttt{getaddrinfo(2)},
dan masing-masing modul membuat soket dengan fungsi \texttt{socket(2)}.
Khusus untuk modul ServerSocket,
soketnya dihubungkan dengan \textit{port} lokal yang telah ditentukan sebelumnya,
dengan fungsi \texttt{bind(2)}.
Soket ini akan menjadi soket sambungan.
Setelah itu ServerSocket mulai mendengarkan dengan fungsi \texttt{listen(2)} sampai ada permintaan sambungan dari \textit{client}.
ClientSocket akan meminta sambungan pada \textit{server} dengan fungsi \texttt{connect(2)},
dan ServerSocket akan menerima permintaan tersebut dengan fungsi \texttt{accept(2)},
dan mengerahkan soket baru untuk komunikasi.
Setelah ini selesai, \textit{client} mulai dapat berkomunikasi dengan \textit{server},
yang dilakukan melalui \textit{method} \texttt{receive} dan \texttt{send} pada tiap modul,
yang masing-masingnya merupakan pembungkus dari fungsi \texttt{recv(2)} dan \texttt{send(2)}.

Demi kejelasan,
untuk meng-\textit{compile} kode yang dibutuhkan untuk pengujian modul ServerSocket,
dari dalam direktori \texttt{server},
perintah-perintah \textit{shell}-nya dapat dipaparkan sebagai berikut.
\begin{enumerate}
\item \verb!g++ -I../common -c ../common/Errors.cpp!
\item \verb!g++ -I../common -c ServerSocket.cpp!
\item \verb!g++ -I../common -c serversockettest.cpp!
\end{enumerate}
Dari sini, ada 3 berkas objek yang dihasilkan,
yaitu \texttt{Errors.o}, \texttt{ServerSocket.o}, dan \texttt{serversockettest.o}.

Begitu juga dengan modul ClientSocket,
yang dari dalam direktori \texttt{client},
perintah-perintah \textit{shell}-nya adalah
\begin{enumerate}
\item \verb!g++ -I../common -c ../common/Errors.cpp!
\item \verb!g++ -I../common -c ClientSocket.cpp!
\item \verb!g++ -I../common -c clientsockettest.cpp!
\end{enumerate}
yang menghasilkan \texttt{Errors.o}, \texttt{ClientSocket.o}, dan \texttt{clientsockettest.o}.

Untuk me-\textit{link} berkas-berkas tersebut,
perintah yang diketikkan adalah\\
\verb!g++ Errors.o ServerSocket.o serversockettest.o -o serversocket!\\
yang akan menghasilkan berkas \textit{executable} bernama \texttt{serversocket}, dan\\
\verb!g++ Errors.o ClientSocket.o clientsockettest.o -o clientsocket!\\
yang akan menghasilkan \texttt{clientsocket}.

Sebuah Makefile (dan juga berkas proyek Xcode) disediakan untuk kenyamanan dalam melalui proses-proses di atas dengan sekali perintah.

Untuk men-\textit{debug},
perintah yang diketikkan adalah \verb!./serversocket! dalam direktori \texttt{server},
dan \verb!./clientsocket! dalam direktori \texttt{client}.


\subsubsection*{DialogueManager}
\hyphenation{men-da-pat-kan di-mo-del-kan}

Kode utama modul ini dapat ditemukan pada berkas \texttt{DialogueManager.cpp} beserta \textit{header}-nya.
Modul ini berkaitan dengan pelaksanaan tugas,
yang nantinya akan dilaksanakan oleh modul TaskManager.
Karena pemenuhan prasyarat-prasyarat pelaksanaan tersebut terkandung dalam percakapan,
pemodelan objek tugas dilakukan pada modul ini.
Dalam pengujiannya, satu kelas tugas sudah tersedia, yaitu tugas membuat surel,
yang kodenya ada pada berkas \texttt{Email.cpp} beserta \textit{header}-nya.
Untuk menguji modul ini, telah disediakan juga berkas \texttt{dialoguemanagertest.cpp}.

Pertama, masukan bagi modul ini yang berupa \textit{dialogue act} ditafsirkan untuk mendapatkan jenisnya dan makna yang terkandung dalam kalimatnya.
Ada tiga dasar dalam menentukan jenis suatu \textit{dialogue act},
yaitu tata bahasa, nada bicara, dan susunan percakapan.
Bagaimanapun, pada tahap awal ini, hanya yang pertama yang sudah mulai diterapkan.
Setelah jenis \textit{dialogue act} pengguna dilacak,
modul ini berusaha untuk menentukan \textit{communicative goal} apa yang diinginkan untuk tercapai dalam percakapan yang sedang berlangsung.
Dalam suatu percakapan, mungkin terdapat lebih daripada satu \textit{communicative goal},
oleh karena itu hal ini dimodelkan dengan \textit{container} pada C++.
Dalam tahap awal, jenis \textit{container} yang dipilih adalah \texttt{stack},
dengan anggapan bahwa \textit{communicative goal} yang terakhir muncul harus dicapai sebelum \textit{communicative goal} yang muncul sebelumnya dapat tercapai.
\textit{Communicative goal} tidak muncul pada setiap \textit{dialogue act},
oleh karena itu modul ini harus melakukan pengolahan juga untuk menentukan apakah ada \textit{communicative goal} yang muncul atau tidak.
Jika ada, maka \textit{communicative goal} tersebut akan ditambahkan ke tumpukan (\textit{stack}) \textit{communicative goals}.
Setelah itu, modul ini akan mulai membuat \textit{dialogue act} tanggapannya,
dengan membuat apa yang disebut sebagai \textit{adjacency pairs}.
Modul ini akan menentukan jenis \textit{dialogue act} apa saja yang cocok dijadikan pasangan bagi jenis \textit{dialogue act} masukan dari pengguna,
yang kemudian kemungkinannya dipersempit lagi dengan menggunakan \textit{information state} sebagai parameter.
Di sinilah modul ini akan menentukan hal seperti apakah PUSPA harus menanyakan tentang suatu informasi kepada pengguna untuk dapat melaksanakan suatu tugas,
dan juga akan melakukan hal-hal yang perlu seperti membuat \textit{instance} dari kelas yang berkaitan dengan tugas yang harus dilaksanakan tersebut (dalam pengujian ini, kelas \texttt{Email}).
Setelah itulah baru modul ini dapat menentukan ``isi'' dari tanggapan yang akan dikeluarkan.

Untuk meng-\textit{compile} kode yang dibutuhkan untuk pengujian modul DialogueManager,
dari dalam direktori \texttt{server},
perintah-perintah \textit{shell}-nya adalah
\begin{enumerate}
\item \verb!g++ -c Email.cpp!
\item \verb!g++ -c DialogueManager.cpp!
\item \verb!g++ -c dialoguemanagertest.cpp!
\end{enumerate}
yang menghasilkan \texttt{Email.o}, \texttt{DialogueManager.o}, dan \texttt{dialoguemanagertest.o}.

Untuk me-\textit{link} berkas-berkas tersebut,
perintah yang diketikkan adalah
\begin{verbatim}
g++ -lpcrecpp Email.o DialogueManager.o dialoguemanagertest.o \
-o dialoguemanager
\end{verbatim}
yang akan menghasilkan \texttt{dialoguemanager}.

Untuk men-\textit{debug},
perintah yang diketikkan adalah \verb!./dialoguemanager! dalam direktori \texttt{server}.
