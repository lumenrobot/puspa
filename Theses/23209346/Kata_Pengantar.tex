\addcontentsline{toc}{chapter}{\uppercase{Kata Pengantar}}
\begin{center}
\large{KATA PENGANTAR}
\end{center}

Alhamdulillahi rabbil ‘alamin. Puji syukur penulis panjatkan ke hadirat Allah SWT atas segala rahmat dan karunia yang dilimpahkan sehingga penulis dapat menyelesaikan tesis ini. Shalawat dan salam tercurah kepada Rasulullah Muhammad SAW beserta keluarganya.

Selama melaksanakan tesis ini, penulis mendapat bantuan dan dukungan dari berbagai pihak. Untuk itu, penulis  mengucapkan terima kasih kepada:

\begin{enumerate}
\item bapak Dr. Ary Setijadi Prihatmanto, ST., MT., selaku Pembimbing I, yang telah memberikan bimbingan dan dorongan semangat selama pelaksanaan tesis, terutama dalam hal penyelesaian produk RESTU;
\item bapak Ir. Tunggal Mardiono, M.Sc., selaku Pembimbing II, yang telah memberikan bimbingan dan perhatian bahkan terhadap hal-hal yang tidak terkait dengan tesis;
\item ibu Dr. Ir. Aciek Ida Wuryandari, MT., bapak Dr. Ir. Hilwadi Hindersyah, M.Sc., dan bapak Dr. Pranoto Hidaya Rusmin, ST., MT., selaku Tim Penguji Sidang Tesis, yang telah memberikan kesempatan bagi penulis
untuk mempresentasikan hasil penelitian dan mengemukakan pendapat;
\item Departemen Pendidikan Nasional atas bantuan Beasiswa Unggulan yang diterima penulis selama menjalani pendidikan program magister;
\item papa, mama, tante, dan kakak-kakak, beserta seluruh keluarga yang senantiasa memberikan semangat dan do’anya;
\item Brenda Ariesty Kusumasari, seorang \textit{supporter} setia dan sahabat seperjuangan untuk 'mengejar' Juli;
\item rekan Andik Taufiq, Nur Ichsan Utama, Erik Prabowo Kamal, Willy Derbyanto, dan Rio Andita Setiabakti (Alm.), tim utama dalam pengembangan RESTU;
\item rekan M. Hakim Adiprasetya dan Syarif Rousyan Fikri, tim \textit{support} dalam pengembangan RESTU;
\item rekan-rekan \textit{Microsoft Innovation Centre} (MIC) dan \textit{Digital Signal Processing Research and Technology Group} (DSP-RTG);
\item rekan-rekan Teknologi Media Digital dan Game (TMDG) angkatan 2009;
\item seluruh staf dan karyawan Laboratorium Sistem Kendali dan Komputer; dan
\item semua pihak yang membantu, yang tidak dapat penulis sebutkan satu persatu.
\end{enumerate}

Penulis menyadari bahwa tesis ini bukanlah tanpa kelemahan, baik dari segi ilmu yang disampaikan maupun teknik penulisannya. Oleh karena itu, penulis mengharapkan adanya kritik dan saran yang dapat disampaikan melalui \textit{email} dengan alamat \href{mailto:aa.nugraha@yahoo.com}{\small\texttt{aa.nugraha@yahoo.com}}. Akhir kata, penulis berharap tesis ini dapat memberikan ilmu dan manfaat bagi yang membacanya.

\vskip 2em

Bandung, Juni 2011

Penulis