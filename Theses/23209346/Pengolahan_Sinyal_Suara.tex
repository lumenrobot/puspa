\section{Gelombang Suara Ucapan Manusia}

Gelombang bunyi adalah gelombang longitudinal yang dihasilkan oleh suatu sumber bunyi dan merambat melalui suatu medium. Cepat rambat gelombang bunyi di udara kira-kira $331.5 + 0.6 T_c \, m/s$, dengan $T_c$ adalah temperatur udara dalam satuan Celcius \cite{huang2001}.

Suara ucapan merupakan gelombang bunyi yang dihasilkan oleh pita suara manusia. Dalam menghasilkan ucapan, pita suara dapat bergetar dan menghasilkan gelombang dengan frekuensi yang disebut sebagai frekuensi fundamental ($F_0$) atau \textit{pitch}. Frekuensi fundamental yang dapat dihasilkan oleh pita suara manusia berbeda antara satu orang dengan orang lainnya. Frekuensi fundamental yang umumnya dapat dihasilkan oleh pita suara manusia adalah sebesar 100 Hz pada pria hingga 250 Hz pada wanita dan anak-anak \cite{rabiner2007}. Sumber lain menyebutkan bahwa frekuensi fundamental yang umumnya dapat dihasilkan adalah sebesar 60 Hz pada pria hingga 300 Hz pada wanita dan anak-anak \cite{huang2001}.

Pada umumnya, energi dari sebuah sinyal berada pada frekuensi fundamental hingga harmonik kesepuluhnya. Oleh karena itu, untuk mendapatkan representasi sinyal suara ucapan manusia yang baik, jangkauan frekuensi yang harus dapat ditangkap adalah $F_0$ hingga $10 \, F_0$ \cite{baken1999}. Sebagai contoh, apabila mengacu pada frekuensi fundamental yang dinyatakan dalam \cite{huang2001}, jangkauan frekuensi yang harus dapat ditangkap agar sinyal suara ucapan baik pria, wanita, mau pun anak-anak dapat direpresentasikan dengan baik adalah 60 Hz hingga 3000 Hz. Di bidang telefoni, kanal suara pada jaringan telepon memiliki lebar pita 4000 Hz (frekuensi 0-4000 Hz), meski transmisi suara hanya memanfaatkan frekuensi dalam jangkauan 300-3300 Hz (lebar pita 3000 Hz) \cite{forouzan2003}.

Berdasarkan teorema Nyquist, untuk memastikan suatu sinyal analog yang dikonversi menjadi sinyal digital dapat direproduksi dengan baik, frekuensi \textit{sampling} (\textit{sampling rate}) sekurang-kurangnya dua kali frekuensi tertinggi dari sinyal analog yang di-\textit{sampling} \cite{forouzan2003, baken1999}. Oleh karena itu, apabila diasumsikan bahwa frekuensi tertinggi dari sinyal suara ucapan manusia adalah 4000 Hz, maka frekuensi \textit{sampling} paling rendah yang boleh digunakan adalah sebesar 8000 Hz.
