\section{Sistematika Pembahasan}

BAB \ref{chap:pendahuluan} menguraikan sejarah singkat dan potensi-potensi pemanfaatan CA, serta proyek pengembangan RESTU, sebuah \textit{engine} ECA, yang kemudian menjadi latar belakang dan motivasi perancangan subsistem penentuan lokasi pengguna berdasarkan sinyal suara. \chaptername~ini juga mendefinisikan batasan-batasan yang digunakan dalam perancangan subsistem tersebut.

BAB \ref{chap:tinjauan_pustaka} membahas landasan teori terkait subsistem penentuan lokasi pengguna berdasarkan sinyal suara yang dirancang dalam penelitian ini. Pembahasan mencakup metode-metode penentuan lokasi sumber suara berdasarkan sinyal suara yang ditangkap.

%\autoref{chap:RESTU} menguraikan arsitektur RESTU secara umum dan menjelaskan bagaimana subsistem yang dirancang diintegrasikan di dalam RESTU.

BAB \ref{chap:perancangan} menjelaskan arsitektur RESTU secara umum dan hubungan subsistem yang dirancang dalam penelitian ini dengan subsistem-subsistem lain yang ada di dalam RESTU, serta menguraikan proses perancangan subsistem penentuan lokasi pengguna berdasarkan sinyal suara, yang meliputi perancangan perangkat keras dan lunak, proses pengambilan dan pengolahan data yang digunakan untuk melatih jaringan syaraf tiruan, serta proses pengujian dan analisis hasil pengujiannya.

BAB \ref{chap:integrasi} membahas proses integrasi subsistem penentuan lokasi pengguna berdasarkan sinyal suara ke dalam RESTU dan proses pengujiannya.

Sedangkan, BAB \ref{chap:kesimpulan} memuat kesimpulan dan saran berdasarkan penelitian yang dilakukan.